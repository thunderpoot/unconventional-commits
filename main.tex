\documentclass[12pt]{article}
\usepackage[utf8]{inputenc}
\usepackage{amsmath}
\usepackage{graphicx}
\usepackage{hyperref}
\usepackage{nopageno}

\title{Unconventional Commits: An Exploration Beyond the Mundane}
\author{
    H. Dog \\
    T. H. Underpoot \\
    P. J. Ox PhD \\
    Department of Physics \\
    Institute for the Study of Advanced Procrastination \\
    \texttt{phys@whatsthat.overthere.edu}
}
\date{\today}

\begin{document}

\maketitle

\begin{abstract}
    This paper presents "Unconventional Commits", a pioneering alternative to the Conventional Commits specification. We examine methods of commit messaging that defy conventional norms, marrying the methodical with the innovative. Our exploration aims to expand the conceptual boundaries of code versioning, asserting the significance of a multifaceted approach to enhancing the development process.
\end{abstract}

\section{Introduction}
    Traditional views on software development often emphasise efficiency and pragmatism, as embodied by the Conventional Commits specification. This perspective, however, neglects the diversity of thought and innovation inherent in the field. "Unconventional Commits" is introduced as a counterpoint to this norm, embracing a broad spectrum of ideas to enrich the narrative of code evolution.

\section{Background}
    \subsection{Conventional Commits}
        The Conventional Commits specification offers a structured framework for commit messages, supporting version control and changelog generation. While it provides clear benefits for automation and clarity, it may limit the scope for creativity and individual expression.

    \subsection{The Perils of Joke Commits}
        Platforms like \textit{What the Commit} demonstrate the use of humour in commit messages, which, while amusing, can detract from the functional and informational value of commit histories. Such practices are discouraged in professional environments due to their potential to undermine the integrity of the development process.

\section{Proposing Unconventional Commits}
    The concept of "Unconventional Commits" introduces novel methods for attaching messages to git commits, emphasising the balance between innovation and integrity. The following methods are proposed:

    \subsection{The Novel Commit}
        This method involves encoding the commit message within the narrative arc of a novel, where the message is revealed through careful analysis of the text. The commit reference directs to a specific passage for interpretation.

    \subsection{Musical Commits}
        Here, commit messages are transcribed into musical compositions, with the notation corresponding to a coded format. The commit provides a recording or sheet music for decoding. MIDI files are also a possibility.

    \subsection{QR Code Image}
        Commit messages are converted into QR codes, then printed, and scanned back into a digital format. The commit includes the image, requiring optical decoding to access the message.

    \subsection{Commit in a Bottle}
        Emulating historical message dissemination methods, commit messages are physically secured in a container and hidden. The commit provides geographical coordinates, inviting a physical search.

    \subsection{Culinary Code}
        Commit messages are embedded within culinary recipes, with each ingredient and cooking step representing elements of the encoded message. The commit challenges the developer to decipher the message through preparation.

    \subsection{Cryptic Crossword Puzzle}
        Commit messages are concealed within the answers to a specially designed crossword puzzle. The commit includes this puzzle, engaging the developer's problem-solving skills to decode the message.

    \subsection{Steganography in an Image}
        This method hides commit messages within images using digital steganography, challenging the developer to employ specific algorithms for extraction.

    \subsection{A Journey Through Software}
        A complex program is developed to reveal the commit message upon execution. The program's source code, obfuscated and intricate, serves as the commit, with execution as the key to decryption.

    \subsection{Video Game Easter Egg}
        A video game contains the commit message, accessible only upon completing challenging tasks or achieving high scores. The commit includes the game, intertwining development with interactive problem-solving.

    \subsection{The Time Capsule}
        Commit messages are physically archived and concealed, with instructions for future retrieval. This method links developers across generations in a shared quest for discovery.

\section{Discussion}
    The introduction of "Unconventional Commits" juxtaposes the conventional with the innovative, highlighting the potential for a diverse array of commit messaging techniques to coexist alongside traditional methods. This exploration not only enriches the software development process but also fosters a culture of creativity and exploration.

\section{Conclusion}
    "Unconventional Commits" presents a novel framework that challenges the traditional confines of commit messaging. By embracing innovative and diverse methods, this approach encourages a reevaluation of the possibilities within software versioning, promoting a multidisciplinary perspective in the realm of version control. The exploration of unconventional methods serves not only to enrich the developer's toolkit but also to inspire a deeper engagement with the process of code evolution, demonstrating that the act of committing code can transcend its utilitarian roots to embrace a broader narrative of creativity and shared human experience.

\begin{thebibliography}{9}
    \bibitem{git}
    Git. \textit{Git Documentation}. \url{https://git-scm.com/doc}

    \bibitem{conventionalcommits}
    Conventional Commits. \textit{Conventional Commits 1.0.0}. \url{https://www.conventionalcommits.org/en/v1.0.0/}

    \bibitem{whatthecommit}
    What the Commit. \textit{What the Commit}. \url{https://whatthecommit.com}

    \bibitem{creativity}
    Stephen P. Blung \textit{The Role of Pizza in Software Development}. Journal of Edible Communication, 1988.
\end{thebibliography}

\end{document}
